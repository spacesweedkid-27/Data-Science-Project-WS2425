% DOCUMENT CLASS %
\documentclass{beamer}

\usepackage[utf8]{inputenc}  % Erlaubt es, Umlaute etc. zu verwenden (Datei muss UTF-8 Kodierung haben)
\usepackage[english]{babel}
\usepackage{hyperref}   % For footnotes

% PACKAGES%
\usepackage{parskip}    % kp warum
\usepackage{amsfonts}   % For a basic mathfont
\usepackage{interval}
\usepackage{amsmath}    % For basic math symbols
\usepackage{bbm}        % For \mathbbm (better version of \mathbb)
\usepackage{csquotes}   % For \textquote{}
%\usepackage{IEEEtrantools}  % For better alignments
%\usepackage{tikz}  % For vectorgraphics
\usepackage{unicode-math}
\usepackage{graphicx}

%% intemize without double spacing
%\usepackage{enumitem}
%\setlist[itemize]{noitemsep}

%% Font config
%\usepackage{fontspec}
%\setmathfont{Latin Modern Math}[Scale=MatchUppercase, FakeBold={2}]

% SETS %
\newcommand*{\R}{\mathbbm R}    % Set of real numbers
\newcommand*{\N}{\mathbbm N}    % Set of natural numbers (beginning at 0)
\newcommand*{\Z}{\mathbbm Z}    % Set of integers
\newcommand*{\Q}{\mathbbm Q}    % Set of rational numbers

\newcommand*{\QDp}{{:}\text{ }} % Macro for ": " so that when writing something like "\forall x:" there is not a separation between the '\forall' and the ':'

% for a funny ¯\_(ツ)_/¯-Emoji
\newcommand{\shrug}[1][]{%
    \begin{tikzpicture}[baseline,x=0.8\ht\strutbox,y=0.8\ht\strutbox,line width=0.125ex,#1]
    \def\arm{(-2.5,0.95) to (-2,0.95) (-1.9,1) to (-1.5,0) (-1.35,0) to (-0.8,0)};
    \draw \arm;
    \draw[xscale=-1] \arm;
    \def\headpart{(0.6,0) arc[start angle=-40, end angle=40,x radius=0.6,y radius=0.8]};
    \draw \headpart;
    \draw[xscale=-1] \headpart;
    \def\eye{(-0.075,0.15) .. controls (0.02,0) .. (0.075,-0.15)};
    \draw[shift={(-0.3,0.8)}] \eye;
    \draw[shift={(0,0.85)}] \eye;
    % draw mouth
    \draw (-0.1,0.2) to [out=15,in=-100] (0.4,0.95); 
    \end{tikzpicture}
}


% pasted content
%       ↓

% https://hartwork.org/beamer-theme-matrix/
\usetheme{Boadilla}
\usecolortheme{dolphin}

% Ganz ohne Hintergrund fand ich das etwas zu plain
\usebeamercolor{block body}
\setbeamercolor{frametitle}{bg=block body.bg}

% Die gängigen Informationen
\institute[\scalebox{0.8}{CAU}]{Christian-Albrechts-Universität zu Kiel}
\author[\scalebox{0.8}{H. Heyden \and M. Materzok \and N. Pulow \and S. Şimşek}]{Henri Heyden \and Mika Materzok \and Nike Pulow \and Senanur Şimşek \\ \small Group \textquote{Für Fortnite}}

\def\pause{\onslide<+>{}}

% https://statisticaloddsandends.wordpress.com/2019/02/18/beamer-inserting-section-slides-before-each-section/
\AtBeginSection[]
{
    \begin{frame}[noframenumbering]
        \frametitle{Overview}
        \tableofcontents[currentsection]
    \end{frame}
}
\AtBeginSubsection[]
{
    \begin{frame}[noframenumbering]
        \frametitle{Overview}
        \tableofcontents[currentsection,currentsubsection]
    \end{frame}
}
\AtBeginSubsubsection[]
{
    \begin{frame}[noframenumbering]
        \frametitle{Overview}
        \tableofcontents[currentsection,currentsubsection]
    \end{frame}
}
%%%%%%%%%%%%%%%%%%%%%%%%%%%%%%%%%%%%%%%%%%%%%%%%%%%%%%%%%%%%%%%%%%%%%%%%%%%%%%%%%%%%%%%%

% macht die footline kleiner
%\setbeamertemplate{footline}{
%    \hspace{0.1cm} \insertshortauthor
%    \hfill
%    \insertshorttitle
%    \hfill
%    \insertframenumber{} / \inserttotalframenumber
%    \hspace{0.1cm}
%}

% entfernt die icons rechts unten
\setbeamertemplate{navigation symbols}{}
